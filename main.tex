\documentclass[9pt]{extarticle}
\usepackage{multicol}
\usepackage{titlesec}
\usepackage[margin=0.5in]{geometry}
\usepackage{fontenc}
\usepackage{csquotes}
\usepackage{verbatim}
\usepackage{float}
\usepackage[dvipsnames]{xcolor}
\usepackage{tcolorbox}
\usepackage{enumitem}
\usepackage[symbol]{footmisc}
\usepackage[T1]{fontenc}
\usepackage{tikz}
\usepackage{xcolor}
\usepackage{fontawesome5} % Icons
\usetikzlibrary{calc}
\usepackage{tabularx}
\usepackage{enumitem}
\usepackage[hidelinks, final]{hyperref}

\definecolor{Headings}{HTML}{577359}
\titleformat{\section}
{\Large\bfseries}
{}
{0em}
{}[\color{Headings}{\titlerule[1.2pt]}]

\titleformat{\subsection}
{\bfseries}
{}
{0em}
{}

\definecolor{grey}{gray}{0.95}
\definecolor{hg}{HTML}{577359}

\setlist[itemize]{leftmargin=*, label={\color{gray}\scriptsize\textbullet}, nosep, itemsep=3pt, topsep=3pt, leftmargin=1em}


 % Counter to track timeline entries
\newcounter{cvcount}

% Command to reset counter at start of section
\newcommand{\resetcvcount}{\setcounter{cvcount}{0}}

% The Dot Command
% 1. Increments counter
% 2. Names the node "tls-<count>" for start
% 3. Names the node "tle" (continuously overwritten) to track the end
\newcommand{\timelinedot}{%
    \stepcounter{cvcount}%
    \begin{tikzpicture}[remember picture, overlay, baseline=-0.7ex]
      \node[circle, fill=hg, inner sep=1pt] (currentdot) {};
        % If first entry, mark as start
        \ifnum\value{cvcount}=1
            \node[coordinate] (tls) at (currentdot) {};
        \fi
        % Mark every entry as potential end (last one wins)
        \node[coordinate] (tle) at (currentdot) {};
    \end{tikzpicture}%
}

% Command to draw the line
% Run this immediately AFTER the table closes
\newcommand{\drawtimeline}{%
    \begin{tikzpicture}[remember picture, overlay]
        \draw[hg, line width=0.5pt] (tls) -- (tle);
    \end{tikzpicture}%
}

% Experience Entry Command
% Args: {Title}{Company}{Bullets}
\newcommand{\cventry}[2]{
    \timelinedot & 
    \textbf{#1} \\    
    & \vspace{-4pt}
      \begin{itemize} #2 \end{itemize} \\ 
    \vspace{4pt} & \\ % Space between jobs
}

\renewcommand{\maketitle}{
	\begin{center}
		{\huge\bfseries HENRY GROVES\\}
		\vspace{.5em}
		% {\Large\bfseries Pontefract New College Student}\\\vspace{.25em}
		{\large hengro44@gmail.com | 07392 072855 | \href{https://github.com/hengrv}{github.com/hengrv} | \href{https://www.linkedin.com/in/hengrv/}{linkedin.com/in/hengrv}}
	\end{center}
}

\renewcommand{\thefootnote}{\fnsymbol{footnote}}

\begin{document}
\thispagestyle{empty}
\maketitle

\vspace{-.2cm}
\begin{minipage}{0.97\textwidth}
	\section{Education}
	\subsection{Further Study, University of Newcastle upon Tyne \hfill 2023-2027}
	\vspace{-.2cm}
	\begin{itemize}
		\item \textbf{M.Sc.+B.Sc.} (Hons) Computer Science
	\end{itemize}
	\vspace{-.4cm}
	\subsection{A Levels, Pontefract New College, Pontefract \hfill 2021-2023}
	\vspace{-0.2cm}
	\begin{itemize}
		\item {\bfseries A Levels:} Mathematics (\textbf{A*}), Further Mathematics (\textbf{A}), Computer Science (\textbf{A}), Physics (\textbf{A})		      \vspace{-.1cm}
		\item {\bfseries EPQ: } ``To what extent will quantum computing replace high performance computing in computational biology?'' (\textbf{A})
	\end{itemize}
	\vspace{-.4cm}
	\normalsize
	% \vspace{-.1 c m}
	\subsection{GCSEs, Carleton High School, Pontefract, West Yorkshire \hfill 2016-2021}
	\vspace{-0.6cm}
	\begin{multicols}{3}
		\begin{itemize}
			% Grades
			\item \textbf{GCSEs:} 5 \textbf{9}'s, 3 \textbf{8}'s
			\item \textbf{BTECs:} \textbf{D*2}, \textbf{D2}
		\end{itemize}
	\end{multicols}

\end{minipage}

\vspace{.4cm}

\begin{minipage}{0.65\textwidth}
	\raggedright

	\section{Work Experience}
	\subsection{\large Northern Powergrid}

	\resetcvcount % Start tracking dots
	% Col 1: Narrow, Centered (for dot). Col 2: Content
	\begin{tabularx}{\textwidth}{@{} >{\centering}p{1.5em} @{} X @{}}

		% JOB 1
		\cventry{\textit{Data Scientist (DSO)}\hfill May-September 2025}
		{
			\item Built a graph-based \textbf{simulation tool} to model high-voltage power networks, in both normal and abnormal conditions.
			\item Used historical and real-time data to simulate network and asset performance with a bottom-up approach, rather than a top-down approach previously used for such analysis.
			\item Tool uses future scenarios to account for the different ways the power grid may evolve looking to 2050.
		}

		% JOB 2
		\cventry{\textit{Data Scientist (DSO)}\hfill May-September 2024}
		{
			\item Designed and architected an end-to-end \textbf{machine learning} solution to predict short-term distribution substation load.
			\item Built a \textbf{robust}, \textbf{efficient} and \textbf{scalable} data pipeline to ingest \textbf{half-hourly} meter readings from $\sim$ 800 substations, filling gaps using AKIMA and lag-based methods.
			\item Used DTW K-Means \textbf{clustering} to find load profiles; proved that models trained on one member of the cluster can effectively predict all members.
			\item Designed and tuned both \textbf{TCN} and \textbf{BiTCN} neural-networks, for each cluster, with carefully selected \textbf{exogenous features} boosting model performance.
			\item Wrote a formal research paper outlining my work and findings to present at \textit{CIRED}.\vspace{-.15cm}		}

		% JOB 3
		\cventry{\textit{Data Analyst (System Forecasting)}\hfill May-August 2023}
		{

			\item Combined numerous data sources with \textbf{SQL} and \textbf{Python} scripting to design and implement a statistical model for predicting the demand of almost \textbf{4 million} customers on low-voltage electrical distribution networks.
			\item Created a dashboard in Excel from the output, and created visualizations for presentations to the wider team and executives.
			\item Extracted and processed data for, and collaborated with other colleagues on regulatory submissions under very tight deadlines whilst also working on the model.
		}
	\end{tabularx}

	\drawtimeline

	\vspace{-.7cm}

	\section{Projects}
	\subsection{``Biggmarket'' - Item swapping app promoting sustainability}
	Built a full-stack item-swapping web application designed to promote sustainable consumption. Built using Next.js, TypeScript, tRPC, Tailwind CSS and backed by a PostgreSQL with Prisma, the app uses a ``swipe left-right'' interface, with real-time user messaging, authentication via Google OAuth, and a comprehensive user review system.
	\vspace{-.15cm}

	\subsection{``CLAuDE'' - Accessible and modular GCM (Global Climate Model)}
	Project manage a large-scale open-source climate model CLAuDE (Climate Analysis using Digital Estimations), written in Python.
	The project's focus is on simplicity and approachability for use as a learning tool.
	Manage contributions, versioning, packaging and community engagement, whilst also writing simulation code and CI/CD pipelines.
	% Provided advice on project structure and version control for the CLAuDE (Climate Analysis using Digital Estimations) GCM, a intuitive and simple model, written in Python and MatPlotLib.
	% It aims to make climate modelling more accessible and provide a simple interface to create custom models.


	% \vspace{-.5cm}




	% \subsection{Northern Powergrid - Investment Planning, Castleford \hfill May 2022}
	% \begin{itemize}
	% 	\item Developed VBA macros in Microsoft Excel to digest financial spend data and create automatic monthly summaries and charts as part of 1 week work experience.
	% \end{itemize}
	% \vspace{-.5cm}



	% \subsection{Hillside Fisheries, Ackworth, Pontefract\hfill June 2021-September 2023}
	% \vspace{-.1cm}
	% \begin{itemize}
	% 	\item Work in the kitchen and on front of house serving and preparing fresh fish and chips. Role includes serving customers, taking payments and using kitchen equipment to prepare food.
	% 	      \vspace{-.1cm}
	% 	\item Developed culinary skills along with conversational and transactional skills. Working in a high pressure environment requires good time management and task prioritisation.
	% \end{itemize}
	% \vspace{-.5cm}
	%



	% \section{Research}
	%
	% \subsection{\textit{Cluster-Based Short Term Load Forecasting for Low Voltage Networks Using Neural and Statistical Models}\hfill CIRED, 2025}
	% H. Groves \& I. Middleton

\end{minipage}
\hspace{0.5cm}
\begin{minipage}{0.3\textwidth}
	\raggedright
	\begin{tcolorbox}[colback=grey]

		\section{Key Skills}

		\begin{itemize}[leftmargin=*]
			\item Problem-Solving
			\item Attention to Detail
			\item Procedural Thinking
			\item Software Development
			\item Machine Learning and AI
			\item Data Exploration \& Analysis

		\end{itemize}

		\subsection{Programming Languages / Frameworks:}
		\begin{itemize}[leftmargin=*]

			\item Python 3
			\item Rust
			\item R
			\item numpy / pandas / scikit
			\item tensorflow / pytorch
			\item matplotlib / seaborn
			\item C/C++
			\item TypeScript
			\item React.JS / Next.JS
			\item Java

		\end{itemize}

		\subsection{Additional Technologies:}
		\begin{itemize}[leftmargin=*]
			\item GNU/Linux
			\item LaTeX
			\item Git / GitHub
			\item SQL (PostGreSQL)
			\item MongoDB
			\item HTML/CSS
			\item tailwindcss
			\item Access / Excel
		\end{itemize}

		\subsection{Languages:}
		\begin{itemize}[leftmargin=*]
			\item English (Native)
			\item German (B1)
		\end{itemize}

	\end{tcolorbox}

\end{minipage}



\end{document}
